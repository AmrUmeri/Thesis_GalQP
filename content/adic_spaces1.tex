\chapter{Adic spaces}
Rings are commutative and have a unit. Complete topological rings are always assumed to be separated.
We introduce \emph{adic spaces}, which is a class of locally ringed spaces.
First we introduce \emph{Huber rings}, \emph{Huber-Tate pairs} and \emph{adic spectra}. A topological ring is called \emph{Huber} if it contains an open subring which is adic for a finitely generated ideal of definition. The original references are \cite{Huber93} and \cite{Huber94}. %Expositions are given in \cite{Morel19}, \cite{SW20}.


\section{Continuous valuation spectrum}
We define the set of continuous valuations on a topological ring. Under an identification, the continuous valuations on the ring form the points of the so-called
\emph{continuous valuation spectrum}.\\

Let $\Gamma$ denote a totally ordered multiplicative abelian group. We write $\Gamma\cup\{0\}$ for the group with some element adjoined and we extend the total order relation and the multiplication to this set by requiring that
 $0<\gamma$ and $0\gamma = \gamma0 =  0$ for all $\gamma\in\Gamma$.

\begin{definition}
Let $A$ be a ring.
A (multiplicative) \emph{valuation} on $A$ is a map
\[\norm{\cdot}\colon A\to\Gamma\cup\{0\}\]
satisfying the following properties:
\begin{enumerate}
\item $\norm{0}=0$, $\norm{1}=1$,
\item $\forall a,b\in A$, $\norm{ab}=\norm{a}\norm{b}$,
\item $\forall a,b\in A$, $\norm{a+b}\leq\max\{\norm{a},\norm{b}\}$.
\end{enumerate}
\vspace{2mm}
The \emph{value group} of a valuation $\norm{\cdot}\colon A\to\Gamma\cup\{0\}$ is the subgroup of $\Gamma$ generated by $\Gamma\cap\norm{A}$.
It will be denoted by $\Gamma_{\norm{\cdot}}$ henceforth. The valuation will be called trivial if $\Gamma_{\norm{\cdot}}=\{1\}$. 
\end{definition}

Notice that the kernel of a valuation $\norm{\cdot}\colon A\to\Gamma\cup\{0\}$, $\ker{\norm{\cdot}}=\{a\in A \:\vert \:\norm{a}=0\}$,   is a prime ideal of $A$. 
Conversely, every prime ideal $\mathfrak{p}\subset A$ gives rise to a trivial valuation on $A$ with kernel $\mathfrak{p}\subset A$.\\

%Consider the following definition:
\begin{definition}
Let $A$ be a ring.
Consider the valuations on $A$,
$\norm{\cdot}_{1}\colon A\to\Gamma_{1}\cup\{0\}$ and 
$\norm{\cdot}_{2}\colon A\to\Gamma_{2}\cup\{0\}$
with value groups 
$\Gamma_{{\norm{\cdot}}_{1}}$, $\Gamma_{{\norm{\cdot}}_{2}}$.
We say that the valuations are \emph{equivalent} if there exists an order-preserving group isomorphism $\varphi\colon\Gamma_{{\norm{\cdot}}_{1}}\xrightarrow{\sim}\Gamma_{{\norm{\cdot}}_{2}}$
with $\varphi\circ{\norm{\cdot}}_{1} = {\norm{\cdot}}_{2}$. We extend the map by requiring $\varphi(0)=0$.
\end{definition}


Valuations define topologies on $A$ in an evident manner. We give the following definition to motivate \emph{continuous valuations}.

\begin{definition}
Let $A$ be a ring and let $\norm{\cdot}\colon A\to\Gamma\cup\{0\}$ be a valuation on $A$. Then the valuation defines a topology on $A$, the \emph{valuation topology},
by declaring a fundamental basis of open subsets around $0$ to be $\{a\in A\:\vert\: \norm{a}<\gamma\}$ for $\gamma\in\Gamma$. This defines a ring topology on $A$.
\end{definition}
\begin{definition}
Let $A$ be a topological ring. Let $\norm{\cdot}\colon A\to\Gamma\cup\{0\}$ be a valuation on $A$. We say that the valuation is \emph{continuous} if for every $\gamma\in\Gamma$,
$\{a\in A\mid \norm{a}<\gamma\}\subseteq A$ is an open subset.
Hence a valuation is continuous if and only if the topology on $A$ is finer than the valuation topology on $A$.
$\cont{A}$ will denote the set of equivalence classes of continuous valuations on $A$. $\cont{A}$ is called the \emph{continuous valuation spectrum} of $A$.
\end{definition}

Let $A$ be a topological ring. For $x\in\cont{A}$ we write $\normx{\cdot}\colon A\to\Gamma\cup\{0\}$ for an arbitrary representative. 
Standard notations are $\mathfrak{p}_x=\ker{\normx{\cdot}}$ and $K_x=A_{\mathfrak{p}_x}/{\mathfrak{p}_x}A_{\mathfrak{p}_x}$ for the associated prime ideal and residue field respectively.
The point $x\in\cont{A}$ is called \emph{analytic} if $\mathfrak{p}_x\subseteq A$ is not open. 
%The point $x\in\cont{A}$ is non-analytic if and only if the quotient topology on $A/\mathfrak{p}_x$ is discrete. 
For any $x\in\cont{A}$ the induced valuation 
$\normx{\cdot}\colon A/\mathfrak{p}_x\to\Gamma\cup\{0\}$ is continuous when $A/\mathfrak{p}_x$ is given the quotient topology. 
By construction the valuation extends to the fraction field
$\normx{\cdot}\colon K_x\to\Gamma\cup\{0\}$. In general analytic points have much better properties than non-analytic points. For example one can show that
for a Huber ring $A$ and analytic point $x\in\cont{A}$ the valuation topology on $K_x$ coincides with the valuation topology coming from a \emph{non-Archimedean} valuation, i.e. $K_x$ is a \emph{non-Archimedean}
field. %This is in fact an equivalence. %, since discrete topologies come from trivial valuations.
The completion of $K_x$ is denoted by $\kappa_x$ in the literature and is called the \emph{completed residue field} 
at $x\in\cont{A}$.
Many geometric objects that we will encounter in the upcoming chapters consist only of \emph{analytic} points. In fact removing non-analytic points of certain valuation spectra turns out to be a useful procedure to construct interesting geometric objects.\\

For certain topological rings $A$ one can topologize $\cont{A}$ such that it becomes a \emph{spectral space}.  %Equivalently it is homeomorphic to the Zariski-spectrum of a ring.% in particular it has a basis of quasi-compact open subsets. 
However this is not the approach taken when defining \emph{adic spaces}. Instead one requires additional properties on the ring $A$ and one is interested in certain subsets of $\cont{A}$.\\

Recall that the underlying set of an affine scheme $\spec{A}$ can be understood as the collection of all ring morphisms $\Hom(A,K)$ to arbtrary fields under a suitable identification. In fact we send a point $x\in\spec{A}$ to 
the residue field $K_x$ and identify field embeddings which make the obvious triangle commute.
%$A\to K_1$, $A\to K_2$ if there exists a morphism $K_1\to K_2$ (or $K_2\to K_1$) 
%if there is an obvious commutative diagram.
 In this sense the continuous valuation spectrum is similar to the Zariski-spectrum.



