\chapter{Perfectoid spaces}
Fix a prime number $p$.
In this chapter we introduce \emph{perfectoid spaces}, which is a full subcategory of the category of analytic adic spaces and which admit a \emph{tilting functor} to perfectoid spaces in characteristic $p$. We also give an exposition of the \emph{tilting equivalence} and the \emph{almost purity theorem}. We will discuss under which assumptions the adic generic fibre, which has been introduced in the previous chapter, is a perfectoid space and determine its \emph{tilt}. This will provide us with additional examples of non-affinoid adic spaces. In later sections we will focus on sheaves for the  pro-\'{e}tale topology, the so-called \emph{pro-\'{e}tale topos} on the category of perfectoid spaces. In particular we will study \emph{diamonds} which we will define as sheaves for the pro-\'{e}tale topology which admit a pro-\'{e}tale cover from a representable sheaf. We state a descent theorem and prove some corollaries.


\section{Perfectoid rings}
In this section we want to study \emph{perfectoid rings}. The main references are \cite{Scholze12}, \cite{Scholzeetcoh21} and \cite{Fontaine13}.
The exposition follows \cite{Morel19} very closely. Fix a prime number $p$.

\subsection{Perfectoid rings: Definition and properties}
We start with the definition of  \emph{perfectoid rings}, heuristically these are uniform Tate rings with enough $p$-power roots. Recall that a Tate ring $A$ is uniform if and only if $A\upcirc$ is a ring of definition of $A$ if and only if
$A\upcirc\subseteq A$ is bounded.

\begin{definition}\label{defperfectoidring}
Let $A$ be a Tate ring. Then $A$ is a \emph{perfectoid ring} if it is complete, uniform and if there exists a pseudo-uniformizer $\pu\in A$ with the following properties:
\begin{enumerate}
\item $p\in\pu^{p}A\upcirc$.
\item The Frobenius map $\varphi\colon A\upcirc/\pu\to A\upcirc/\pu^p$, $a\mapsto a^p$, is bijective.
\end{enumerate}
\end{definition}
In particular, the above Frobenius map is a ring isomorphism.
Notice also that if $A$ is a perfectoid ring of characteristic $0$, then $p$ is topologically nilpotent.
Hence if $p$ is a unit in $A$, then the topology on $A\upcirc$ is the $p$-adic topology and $A=A\upcirc[\frac{1}{p}]$.
%by corollary \ref{cortatering1}.\\

The next lemma shows that it is enough to assume that the Frobenius map is surjective in the above definition.

\begin{lemma}
Let $A$ be a Tate ring and assume we have a pseudo-uniformizer $\pu\in A$ such that $p\in\pu^{p}A\upcirc$. Then the Frobenius map 
$\varphi\colon A\upcirc/\pu\to A\upcirc/\pu^p$ is injective.
\end{lemma}
\begin{proof}
Let $a\in A\upcirc$ such that $a^p\in\pu^{p}A\upcirc$. Then $\frac{a^p}{\pu^p}\in A\upcirc$, hence $\frac{a}{\pu}\in A\upcirc$ since $A\upcirc$ is integrally closed in $A$.
Thus $a\in\pu A\upcirc$.
\end{proof}

We now want to prove that the definition of a perfectoid ring does not depend on the choice of pseudo-uniformizer. The following lemma makes this precise:
\begin{lemma}\label{perfringequivdeflemma}
Let $A$ be a complete and uniform Tate ring. Assume we have a pseudo-uniformizer $\pu\in A$ with $p\in\pu^pA\upcirc$.
Then we have the following result:\\
The Frobenius $A\upcirc/\pu A\upcirc\to A\upcirc/\pu^pA\upcirc$ is surjective if and only if the Frobenius\\ $A\upcirc\to A\upcirc/pA\upcirc$ is surjective.
In particular this means that $A$ is a perfectoid ring if and only if every element in $A\upcirc/pA\upcirc$ is a $p$-th power.
\end{lemma}
\begin{proof}
For the easy direction, let $a\in A\upcirc$. Then we can find $b\in A\upcirc$ with $a-b^p\in pA\upcirc\subseteq\pu^pA\upcirc.$ 
For the other direction, we need the following successive approximation lemma:
There exists a sequence $(a_n)_{n\geq0}$ of elements in $A\upcirc$ such that for every $n\geq0$, $a-\sum_{i=0}^{n}a_i^p\pu^{pi}\in\pu^{p(n+1)}A\upcirc$. Hence $a=\sum_{i\geq0}a_i^p\pu^{pi}$.
By induction and the binomial formula one shows: For every $n\geq0$, $\sum_{i=0}^{n}a_i^p\pu^{pi}-(\sum_{i=0}^{n}a_i\pu^{i})^p\in pA\upcirc$. Hence $a-(\sum_{i=0}a_i\pu^{i})^p\in pA\upcirc$.
\end{proof}


We needed the following successive approximation lemma for the previous proof, which we want to state and prove now:

\begin{lemma}
Let $A$ be a Tate ring and let $\pu\in A$ be a pseudo-uniformizer such that $p\in\pu^{p}A\upcirc$. Let $a\in A\upcirc$ and suppose that the Frobenius $A\upcirc/\pu A\upcirc\to A\upcirc/\pu^pA\upcirc$ is surjective.
Then there exists a sequence $(a_n)_{n\geq0}$ of elements in $A\upcirc$ such that  for every $n\geq0$,
 $a-\sum_{i=0}^{n}a_i^p\pu^{pi}\in\pu^{p(n+1)}A\upcirc$. 
If $A$ is separated and uniform, we can write $a=\sum_{i\geq0}a_i^p\pu^{pi}$.
\end{lemma}
\begin{proof}
This is by induction on $n\geq0$. By assumption we can find $a_0\in A\upcirc$ with $a-a_0^p\in\pu^pA\upcirc$.
Assume we have found $a_0\dots a_{n-1}\in A\upcirc$ such that $a-\sum_{i=0}^{n-1}a_i^p\pu^{pi}\in\pu^{pn}A\upcirc$. Let $b\in A\upcirc$ with $a-\sum_{i=0}^{n-1}a_i^p\pu^{pi} = \pu^{pn}b$.
By assumption we can find $a_n\in A\upcirc$ such that $b-a_n^p\in\pu^pA\upcirc$.
Hence $a-\sum_{i=0}^{n}a_i^p\pu^{pi}\in\pu^{p(n+1)}A\upcirc$.
The last claim uses the fact that the topology on $A\upcirc$
is the $\pu$-adic topology.
\end{proof}

Before continuing with perfectoid rings, we state and prove a lemma, which is used in many key constructions. We will refer to it as key lemma.
Let $S$ be any ring. We can consider the Frobenius as a multiplicative map $S\to S$, $s\mapsto s^p$. Iterating Frobenius gives a projective system. The projective limit is only a multiplicative monoid in general, however it is a ring if the Frobenius map is a ring morphism.

\begin{lemma}\label{keylemma}
Let $S$ be a ring, $\pu\in S$ an element with $p\in\pu S$ and $S$ is complete for the $\pu$-adic topology. Then we have an isomorphism of multiplicative monoids induced by the canonical map $S\to S/\pu S$,
\[\varprojlim_{s\mapsto s^p}S\xrightarrow{\sim}\varprojlim_{\ovl{s}\mapsto\ovl{s}^p}S/\pu S,\]
\[(s_n)_{n\geq0} \mapsto (\ovl{s_n})_{n\geq0}.\]
In particular there is a perfect ring structure of characteristic $p$ on $\displaystyle\varprojlim_{s\mapsto s^p}S$.
\end{lemma} 
\begin{proof}
First we construct a multiplicative map: 
\[\displaystyle\varprojlim_{\ovl{s}\mapsto\ovl{s}^p}S/\pu S\to S.\]
Let $ (\ovl{s_n})_{n\geq0}\in\varprojlim_{\ovl{s}\mapsto\ovl{s}^p}S/\pu S$ and choose arbitrary lifts
$s_n\in S$, $n\geq0$. Notice that by construction we have $s^p_{n+1}-s_n\in\pu S$.
This implies $s^{p^{n+1}}_{n+1}-s^{p^n}_n\in\pu^{n+1}S$ by induction and the binomial formula. In particular $(s_n^{p^n})_{n\geq0}$ is a Cauchy sequence and converges to an element in $S$, which we will denote by $\lim_{n\to\infty}s_n^{p^n}$.
One shows that the limit does not depend on the choice of lifts, 
meaning that if $s_n-t_n\in\pu S$, then $s_n^{p^n}-t_n^{p^n}\in\pu^{n+1}S.$ This then defines the map in an evident way:
\[f\colon\displaystyle\varprojlim_{\ovl{s}\mapsto\ovl{s}^p}S/\pu S\to S, \:\:(\ovl{s_n})_{n\geq0}\mapsto \lim_{n\to\infty}s_n^{p^n}.\]
We then define the multiplicative map, which is an inverse to the map stated in the lemma:
\[\varprojlim_{\ovl{s}\mapsto\ovl{s}^p}S/\pu S\rightarrow\varprojlim_{s\mapsto s^p}S,\:\: (\ovl{s_n})_{n\geq0} \mapsto (f((\ovl{s_{n+r}})_{n\geq0}))_{r\geq0}.\]
One checks that this indeed defines the inverse. The isomorphism does not depend on the choice of $\pu\in S$.
Notice that if $(\ovl{s_n})_{n\geq0}\in\varprojlim_{\ovl{s}\mapsto\ovl{s}^p}S/\pu S$, then this element has a $p$-th root, namely we just choose
$(\ovl{s_{n+1}})_{n\geq0}\in\varprojlim_{\ovl{s}\mapsto\ovl{s}^p}S/\pu S$. In particular $\varprojlim_{\ovl{s}\mapsto\ovl{s}^p}S/\pu S$ is perfect.

\end{proof}

We used the following basic lemma for the proof:
\begin{lemma}
Let $S$ be a ring, $\pu\in S$ with $p\in\pu S$. Let $a,b\in S$ with $a-b\in\pu S$. Then $a^{p^n}-b^{p^n}\in\pu^{n+1}S$ for $n\geq 0$.
\end{lemma}
\begin{proof}
It is enough to prove that if $a-b\in\pu^n S$, then $a^p-b^p\in\pu^{n+1} S$.  
Let $a=b+\pu^nc$ for some $c\in S$ and apply the binomial formula.
\end{proof}


We now want to describe explicitely the additive structure on $\varprojlim_{s\mapsto s^p}S$. Consider the following:\\

Let $(a_n)_{n\geq0}, (b_n)_{n\geq0}\in\varprojlim_{s\mapsto s^p}S$. Clearly $a_n+b_n\in S$ is a lift of $\ovl{a_n+b_n}\in S/\pu S$.
Hence we have the following addition rule:
\[(a_n)_{n\geq0}+(b_n)_{n\geq0} = (\lim_{k\to\infty}(a_{k+n}+b_{k+n})^{p^k})_{n\geq0}\in\varprojlim_{s\mapsto s^p}S.\]
For each ${n\geq0}$, $((a_{k+n}+b_{k+n})^{p^k})_{k\geq0}$ is a Cauchy sequence.\\


We can describe perfectoid rings of characteristic $p$ in the following way:
\begin{lemma}\label{perfringplemma}
Let $A$ be a Tate ring of characteristic $p$. Then $A$ is perfectoid if and only if $A$ is complete and perfect.
\end{lemma}
\begin{proof}
The difficult part is to prove that perfect complete Tate rings of characteristic $p$ are uniform.
This is a technical result, see for example \cite[theorem IV.1.3.5]{Morel19}.
For the proof of the lemma see \cite[proposition V.1.1.8]{Morel19}.
\end{proof}










